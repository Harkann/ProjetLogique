\documentclass{report}
\usepackage{minted}
\usepackage[utf8]{inputenc}
\usepackage[T1]{fontenc}
\usepackage{amsmath}
\usepackage{amssymb}
\title{Projet Logique\thanks{Chargé de projet : François THIRE}}
\author{Michaël PAULON}
\date{L3 informatique ENS Cachan 2016-2017}
\begin{document}
\maketitle
\tableofcontents

\chapter{Introduction}

\section{Langages utilisés}

\paragraph{Python3.5} pour la résolution de problèmes

\paragraph{C} pour le SAT solveur

\section{Problèmes traités}
- Sudoku

\chapter{Solveurs}

\section{Structure du projet}

\subsection{Solveur de Sudoku}

\paragraph{Fonctionnement}:\\
- Transcrit le Sudoku en formule CNF au format DIMACS.\\
- Exécute le SAT solveur sur la formule générée.\\
- Récupère la sortie du solveur et la convertie en sudoku résolu.\\
- Ecrit cette solution dans ``sudoku.solved''

\paragraph{Fichiers}:\\
- sudoku.py

\subsection{SAT Solveur}

\paragraph{Fonctionnement}:\\
- Parse la formule DIMACS. \\
- Applique l'algorithme DPLL sur la formule. \\
- Si la formule est satifiable, renvoie SAT et une instance la satisfaisant. \\
- Sinon renvoie UNSAT.

\paragraph{Fichiers}:\\
- parseur.c / parseur.h \\
- solveur.c / solveur.h \\
- main.c

%\subsection{Script de benchmark}

\section{Utilisation}

\subsection{Compilation}
\begin{minted}{bash}
make all
\end{minted}

\subsection{Pour tester le SAT solveur}

\paragraph{Input}
Un fichier contenant une formule CNF au format DIMACS.

\paragraph{Commandes}
\begin{minted}{bash}
make all
./SATsolveur $(cat lefichierCNF)
\end{minted}

\paragraph{Résultat}
Produit un fichier ``cnf.solved'' contenant : \\
- SAT ou UNSAT \\
- si SAT : le status des différentes variables.

\subsection{Pour tester le solveur de Sudoku}

\paragraph{Input}
Un fichier contenant le sudoku au format suivant :
\begin{minted}{bash}
0 0 0|0 0 0|0 0 0
0 0 0|0 0 0|0 0 0
0 0 0|0 0 0|0 0 0
0 0 0|0 0 0|0 0 0
0 0 0|0 0 0|0 0 0
0 0 0|0 0 0|0 0 0
0 0 0|0 0 0|0 0 0
0 0 0|0 0 0|0 0 0
0 0 0|0 0 0|0 0 0
\end{minted}
Où 0 représente une case vide.

\paragraph{Commandes}
\begin{minted}{bash}
make all
python sudoku.py lefichierSudoku
\end{minted}

\paragraph{Résultat}
Produit un fichier ``sudoku.solved'' contenant le sudoku résolu et l'affiche sur STDOUT.


\chapter{Performances}

\end{document}